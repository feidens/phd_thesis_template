%
%%%%%%%%%%%%%%%%%%%%%%%% Environments
%
% use them with 
% \begin{...}
% \end{...}





%%% Theorem environment definitions
\declaretheoremstyle[%
					spaceabove = \baselineskip,%
					spacebelow = \baselineskip,%
					bodyfont = \normalfont,%
					headfont = \normalfont\bfseries,% 
					postheadspace = \newline,%
					notebraces = {(}{)},%
					notefont = \normalfont,%
					headpunct = {},
					headformat = \NAME\ \NUMBER \NOTE,
					]{mytheoremstyle}
\declaretheorem[%
				name = Definition,% 
				parent = section,% 
				style = mytheoremstyle,% 
				qed = {$\diamond$},%
				]{definition}
\declaretheorem[%
				name = Construction,% 
				parent = section,% 
				style = mytheoremstyle,% 
				qed = {},%
				]{construction}

\declaretheoremstyle[%
					spaceabove = \baselineskip,%
					spacebelow = \baselineskip,%
					bodyfont = \normalfont,%
					headfont = \normalfont\bfseries,% 
					postheadspace = { },%
					notebraces = {}{},%
					notefont = \normalfont,%
					headpunct = {},%
					headformat = \NAME\ \NUMBER:,%\NOTE,
					]{myremarkstyle}
\declaretheorem[%
				name = Remark,%
				parent = section,%
				style = myremarkstyle,% 
				qed = {},%
				]{remark}
\declaretheorem[%
				name = Example,%
				parent = section,%
				style = myremarkstyle,% 
				qed = {},%
				]{example}
							
				

%\newtheoremstyle{definition}% name
%{\baselineskip}% Space above
%{\baselineskip}% Space below
%{\normalfont}% Body font
%{}% Indent amount
%{\normalfont\bfseries}% Theorem head font
%{}% Punctuation after theorem head
%{\newline}% Space after theorem head
%{}% Theorem head spec (can be left empty, meaning normal)
%\theoremstyle{definition}
%\newtheorem{definition}{Definition}[section]
%\newtheorem{construction}{Construction}[section]
%\newtheorem{example}{Example}[section]
\theoremstyle{plain}
\newtheorem{theorem}{Theorem}[section]
\newtheorem{lemma}{Lemma}[section]
\newtheorem{corollary}{Corollary}[section]

%%%%%%%%%%%%%%%%%%%%%%%%%%%%%%%%%%%%%%%%%%


% === Experiment env ===
\DeclareFloatingEnvironment[fileext=lop,placement={tbh!},name=Experiment,within=chapter]{experiment}
\captionsetup[experiment]{labelfont={},name={Experiment},labelsep=colon}

\makeatletter
\let\c@experiment\c@figure
\makeatother



\newenvironment{keywords}%
{%
	\begin{trivlist}%
		\item[]{\bfseries Keywords:}%
	}%
	{%
	\end{trivlist}%
}%


%%%%% old ABS environments

%
%\newenvironment{experiment}[1][]{%
%		\begin{list}{}{%
%			\leftmargin=1em%
%			\parsep=0pt%
%			\itemsep=3pt%
%			}%
%			\ifx&#1&%
%				% #1 is empty
%			\else%
%				\vspace{0.15em}
%				\item[#1]~
%				
%			\fi%				 	
%		
%		}{%
%		\end{list}%
%}%


%\newenvironment{experimentcentered}[2][0.5]{%
%	\begin{center}
%	\begin{varwidth}{\textwidth}
%		\begin{list}{}{%
%			\leftmargin=1em%
%			\parsep=1pt%
%			\itemsep=3pt%
%			}%
%				%\vspace{100.15em}
%				\item[\underline{
%						\makebox[#1\textwidth][l]{#2}}]~		 	
%		}{%
%		\end{list}%
%		~
%		\bigskip
%	\end{varwidth}	
%	\end{center}
%}%



%%%%% Experiments in cryptocode

%\createpseudocodeblock{pcb}{center , boxed}{}{}{}
%\createprocedureblock{expproc}{center , boxed}{}{$\mathrm{Experiment}$\xspace}{ linenumbering}
% this is what I used:
\createprocedureblock{experimentcentered}{}{}{}{ linenumbering, codesize = \normalsize }




%%%% Style %%%%%

% local table of content
\newcommand{\localtoc}[1]{
  %\rule{\linewidth}{1pt}
  \textbf{Summary.}
  #1 % this is the argument for the summary
  \etocsettocstyle{
  		{\color{myorange}\bigskip\hrule\kern0pt\hrule\medskip\smallskip}
  	}
  		{\noindent{\color{myorange}\hrule\kern0pt\hrule}\bigskip\par\noindent}
  \localtableofcontents
}

% style of chapter pages
\usepackage{scrlayer-scrpage}
\KOMAoptions{%
	headwidth=(\textwidth+.5\marginparwidth):0mm,% width of header
	chapterprefix=false,
}

\pagestyle{scrheadings}% activate the style
\clearpairofpagestyles
%\ohead[]{\thepage}% pagenumber in outer margin
%\ofoot{\pagemark}
\ofoot[\pagemark]{\pagemark} 
%\cfoot[\pagemark]{\pagemark} 
\lohead[]{\rightmark}% section name in odd pages
\rehead[]{\leftmark}% chapter name in even pages
%\setkomafont{pageheadfoot}{\bfseries}% all headers and footer in bold font


% Part style

\renewcommand{\partheadmidvskip}{%
  \par\nobreak\vspace*{.5\baselineskip}%
  {
  \color{myorange}
	    	\kern 10pt \hrule width \hsize height 1pt \kern 7pt 
  }
  \par\nobreak\vspace*{.5\baselineskip}%
}
\renewcommand{\partpagestyle}{empty}

%\newlength{\currenttitlewidth}%
%\newlength{\currentnumberwidth}%
%\renewcommand{\chapterlinesformat}[3]{
%	\begin{tikzpicture}[remember picture,overlay]
%	\coordinate (meet) at ($(current page marginpar area.north east) +
%	(-\dimexpr0.3\marginparsep+\marginparwidth\relax, -3\baselineskip)$);
%	\ifx#2\empty %nothing to do
%	\else
%	\node[anchor = south west] (num) at (meet) {\resizebox{!}{2\baselineskip}{\textcolor{gray}{\thechapter}}};
%	\fi
%	\node [anchor = south east, align=right] (title) at (meet)
%	{\begin{varwidth}{\textwidth}\raggedleft #3\end{varwidth}};
%	\draw (title.south west) -- ($(title.south east -|current page marginpar area.east) - (1.5ex,0)$);%
%	\end{tikzpicture}
%}
%\renewcommand{\chapterlinesformat}[3]{%
%%\parbox[b]{\dimexpr\textwidth-2em\relax}{\raggedchapter #2 #3}
%	\makebox[0.99\textwidth][b]{% dont know why but 1.0\textwidth causes badboxes
%				\ifx#2\empty%
%			% nothing to do when no chapter number is set
%		\else%
%			\settowidth{\currentnumberwidth}{% compute width of number
%				\resizebox{!}{1.8\baselineskip}{\thechapter}%
%			}%
%				\resizebox{!}{1.8\baselineskip}{\thechapter}% set the number
%		\fi%
%		%
%		\hspace*{\dimexpr+8pt\relax}
%		\parbox[b]{\textwidth}{% set the title
%			
%			\raggedright #3%
%		}%
%	}%
%	% draw the line
%	\vskip-.2\baselineskip% small vertical skip
%	\settowidth{\currenttitlewidth}{% compute length of line
%		\begin{varwidth}{\textwidth}%
%			\raggedleft#3%
%		\end{varwidth}%
%	}%
%	\ifdim\textwidth<\currenttitlewidth%
%		\setlength{\currenttitlewidth}{\textwidth}%
%	\fi%
%	\hspace*{\dimexpr-\marginparwidth+17pt\relax}%
%	\makebox[0pt][l]{%
%		\color{myorange}\rule{\dimexpr\currenttitlewidth+\marginparwidth-17pt\relax}{1pt}%
%	
%	}%
%}



\renewcommand*\chapterformat{%
    \quad
    \textcolor{myorange}{\fontsize{80pt}{80pt}\selectfont\thechapter}%
}

\renewcommand*\chapterlinesformat[3]{%
    \parbox[b]{\dimexpr\textwidth-2em\relax}{\raggedchapter #3}%
    \hfill
    \makebox[0pt][l]{#2}%
    %\bigskip
    \newline
    \makebox[0pt][l]{%
		\color{myorange}\rule{\dimexpr\textwidth+\marginparwidth\relax}{1pt}%
	}
}

